\documentclass[11pt]{amsart}
\usepackage{amsmath}
\usepackage{amssymb}
\usepackage{amsthm}
\usepackage{graphicx}
\usepackage{algorithmic}
\usepackage{multicol}


\newcommand{\R}{\mathbb{R}}

\newtheorem{thm}{Theorem}[section]
\newtheorem{lem}[thm]{Lemma}

\theoremstyle{remark}
\newtheorem{rem}{Remark}
\newtheorem{defn}{Definition}


\title{TV-Framelet regularization for MRI reconstruction in the
presence of field inhomogenities.}
\author{Ryan Compton}
\date{October 26, 2010}


\begin{document}

\begin{abstract}

TODO: framelet talk, inhomogeneity talk, bregman talk (like in JF Cai's paper), compare with TV? Why is TV sucking?ahha! The framelet term improves the condition number of the pcg! ha! yep. that's the main reason to do this. dang.

\end{abstract}

\maketitle

\section{Introduction}

Rapid acquisition of MR images via reconstruction from undersampled $k$-space data has the potential to greatly decrease MRI scan time on existing medical hardware. To this end iterative image reconstrction based on the technique of compressed sensing has become the method choice for many researchers \cite{Lustig2007}. However, while conventional compressed sensing relies on random measurements from the discrete Fourier transform of our image, actual MR scans often suffer from off-resonance effects and thus generate data by way of a non-Fourier operator, complicating image reconstruction methods and introducing additional computational bottlenecks \cite{Fessler2005}.

Our model of MR image creation requires that we solve for the proton density map, $\rho(\mathbf{r})$, in the signal equation

\begin{equation}\label{sigeq}
s(t) = \int_{\mathbb{R}^2} \rho(\mathbf{r})e^{-z(\mathbf{r})t}e^{-2\pi i \mathbf{k}(t) \cdot \mathbf{r}} d\mathbf{r} + \epsilon
\end{equation}
where $s(t)$ is the recorded signal, $z(\mathbf{r})$ contains information about relaxation and off-resonance effects, the $k$-space coordinates $\mathbf{k}(t)$ are known only on a nonuniform and possibly undersampled grid and $\epsilon$ is receiver noise. Discretizing the physical space integral in (\ref{sigeq}) at $n$ points and collecting $m$ samples in $k$-space leads to the $m \times n$ linear system

\begin{equation}\label{dissigeq}
\vec{s} = A \vec{\rho} + \vec{\epsilon}
\end{equation}
with sample matrix
\begin{equation}
A_{ij} = e^{-z(\mathbf{r}_j)t_i}e^{-2\pi i \mathbf{k}(t_i) \cdot \mathbf{r}_j}.
\end{equation}


A straightforward approach to this discrete reconstruction problem (\ref{dissigeq}) is infeasible for practical image sizes. For example, in order to produce a $512 \times 512$ image the (dense) $A$ matrix takes on dimensions $512^2 \times 512^2$. Storing $A$ using 32-bit floating point precision results in a matrix requiring 256GB of space, prohibitively large on most current machines. Operator compression is thus necessary before attempting to solve the linear system.

Ignoring the $z(\mathbf{r})$ factor and filling $k$-space completely on a uniform grid yields
\begin{equation}
A = \mathcal{F}
\end{equation}
the discrete Fourier matrix. We may determine $\rho$ here by simply inverting a single Fourier transform.

Failing to accomodate for off-resonance effects leads to blurring and image distortion unless the background field is perfectly homogenous while filling $k$-space completely increases scan time beyond what is needed for exact image reconstruction.

We reduce scan time by appealing to the theory of compressed sensing and downsample our Fourier matrix by multiplication with a random $m \times n$ matrix $R_{ij} = 1 \text{ if we record } k_j \text{, } 0 \text{ else}$. Thus
\begin{equation}
A = R\mathcal{F}
\end{equation}
and we account for the missing data by taking advantage of the fact that our image is sparse in some transform domain by introducing $l1$ regularization. The canonical image reconstruction problem is then
\begin{equation}\label{csequ}
\underset{\rho}{\operatorname{argmax}} \: J(\rho)  \text{ subject to } R\mathcal{F} \rho = s
\end{equation}
where $J$ represents some form of $l1$ regularization. Perfect image reconstruction is possible as $R\mathcal{F}$ satifies the restricted isometry property.

Our model follows the work of [] [] [] taking into account the phase accural due to off-resonant frequency by approximating $A$ with a combination of nonuniform Fourier transforms. That is,
\begin{equation}
A = \sum_{l=1}^L B_l \mathcal{G} C_l 
\end{equation}
where $B_l$ and $C_l$ are diagonal matrices used in a rank $L$ approximation of $e^{-z(\mathbf{r})t}$ and $\mathcal{G}$ is a nonuniform discrete Fourier transform of type 2 []. While this generalized model hurts our restricted isometry constants image recovery from undersampled $k$-space is still possible when enough data is collected.

The optmization we solve is
\begin{equation}\label{gcsequ}
\underset{\rho}{\operatorname{argmax}} \: J(\rho)  \text{ subject to } A \rho = s.
\end{equation}
In this work we advocate a hybrid of total vairation and framelet regularizers,
\begin{equation}
J(\rho) = | \nabla \rho| + | D \rho |
\end{equation}
where the first term is a total vairation norm and $D$ is the discrete framelet decomposition []. As we will see, not only does this combined regularization improve image quality, it markedly decreases computational cost when compared with the standard total vairation regularizer as the subproblems we solve in our minimization become better conditioned.

\section{Method}

\section{Old stuff}


using the Split Bregman method []. This iterative method gives fast image recovery as well as flexibilty with regard to choice of $J$. In this work we advocate the regl



Initial research was done with $J(\rho) = | \nabla \rho|$, the Total Variation norm. Researchers have found improved image quality in this setting by combining Total Vairation with a Besov regularizer 
$J(\rho) = | \nabla \rho| + | W \rho |$ where $W$ is the discrete Haar wavelet transform.

Exact reconstruction of the image is possible from few samples as $R\mathcal{F}$ satisfies the ``restricted isometry propery'' \cite{Candes2006}. Solving the optimization can be done using only applications of $R\mathcal{F}$ which requires no storage space.

Varying our transform domain may result in improved image quality. Aside from the tradional least squares approach, regularizations based on, wavelets \cite{Lustig2007} and nonlocal TV have already been proposed. 
 







Efficient methods for fast application of the Fourier integral operator with general $\phi(\mathbf{r},t)$ are an area of active research \cite{Cand2006}. 


\section{Theory}

Standard clinical MRI scanners produce a powerful (1.5-3 T) static background field, $\mathbf{\hat{B}}_0(\mathbf{r})$,
to induce a net magnetization, $\mathbf{M(\mathbf{r})}$, on the sample being imaged.

Within a few seconds $\mathbf{M}(\mathbf{r})$ aligns in equilibrium with
the background field $\mathbf{\hat{B}}_0(\mathbf{r})$. In equilbrium the
magnitude of the net magnetization is dependent on the substance being
imaged and is proportional to $B_0$ and the local density of protons, $\mathbf{\rho}(\mathbf{r})$.


To spatially encode data gradient coils produce weak (450 gauss vs the
15,000 gauss background field) time varying magnetic fields
$\mathbf{B}_G(\mathbf{r},t)$ which are added to
$\mathbf{B}_0(\mathbf{r})$. These fields are designed to tip
$\mathbf{M}$ into the plane perpendicular to the level sets of
$\mathbf{\hat{B_0}}$. Traditional scanners produce a static field along
the $z$ direction only, $\mathbf{\hat{B}}_0(\mathbf{r}) =
B_0(\mathbf{r})\mathbf{z}$ allowing us to select slices with
constant $z$ and image within a slice by tipping
$\mathbf{M}$ into the $xy$-plane.


In the high field regime \cite{Kelso2009} we neglect concomitant fields and project the components of the gradient fields along $\mathbf{z}$ so our encoding field becomes
\begin{equation}
\mathbf{B}(\mathbf{r},t) = \mathbf{\hat{B}}_0 + \mathbf{B_G}(\mathbf{r},t)\cdot \mathbf{\hat{B}}_0 \frac{\mathbf{\hat{B}}_0}{||\mathbf{\hat{B}}_0||^2}
\end{equation}


In a traditional MR scanner, the gradient fields depend linearly with position in each slice:
\begin{equation}
\mathbf{B}(\mathbf{r},t) = \mathbf{\hat{B}}_0(\mathbf{r}) + (x,y,z)\cdot(G_x(t),G_y(t),G_z(t))\mathbf{z}
\end{equation}


The frequency of precession is governed by the Larmor equation
\begin{equation}
\omega(\mathbf{r},t) = \gamma B_z(\mathbf{r},t)
\end{equation}

Describing the magnetic field in the transverse plane as $M(\mathbf{r},t) = M_x + M_yi$ we model the time evolution of each voxel as
\begin{equation}
M(\mathbf{r},t) = M(\mathbf{r},0) e^{-t/T_2(r)}e^{-i\int_0^t(\omega(\mathbf{r},t')dt'}
\end{equation} 
where the $T_2(\mathbf{r})$ relaxation factor accounts for the loss of net magnetization due to the samples magnetic field and $\omega(\mathbf{r},t) = \gamma \mathbf{B}(\mathbf{r},t)$.


Making use of Faraday's Law in a receiver coil yields the signal equation

\begin{equation}\label{sigeq}
s(t) = \int_{\mathbb{R}^2}c(\mathbf{r})\rho(\mathbf{r})e^{-t/T_2(r)}e^{-i\phi(\mathbf{r},t)}d\mathbf{r}
\end{equation}

where $c(\mathbf{r})$ is the response pattern of the receiver coil, $\rho(\mathbf{r}) = M(\mathbf{r},0)$ is the proton density map, and $\phi(\mathbf{r},t) = \int_0^t \omega(\mathbf{r},t') - \omega_0 dt'$. The constant $\omega_0$ provides demodulation around the center frequency.


The goal of MR image reconstruction is to solve for $\rho(\mathbf{r})$ in (\ref{sigeq}).



Discretizing (\ref{sigeq}) and row major flattening the 2d arrays yields the linear system

\begin{equation}\label{dissigeq}
\vec{s} = A \vec{\rho} + \vec{\epsilon}
\end{equation}

\begin{equation}
A_{ij} = c(\mathbf{r}_j)e^{-t_i/T_2(\mathbf{r}_j)}e^{-i\phi(\mathbf{r}_j,t_i)}
\end{equation}

where $\vec{\epsilon}$ accounts for receiver noise.


The discrete reconstruction problem (\ref{dissigeq}) is often underdetermined, nonuniformly sampled in $k$-space, and very large. For a $512 \times 512$ image the (dense) $A$ matrix takes on dimensions $512^2 \times 512^2$. Using 32-bit floating point precision results in a matrix requiring 256GB of RAM.

Operator compression is thus neccessary before one may invert the linear system. Efficient
methods for fast application of the Fourier integral operator with
general $\phi(\mathbf{r},t)$ are an area of active research \cite{Cand2006}. 

When given uniformity along $\mathbf{z}$ in the background field we write
\begin{equation}
\mathbf{B}(\mathbf{r},t) = (B_0 + \Delta B_0(\mathbf{r}) + \mathbf{G}(t)\cdot \mathbf(r))\mathbf{z}
\end{equation}
to obtain
\begin{equation}
e^{-i\phi(\mathbf{r},t)} = e^{-i\Delta \omega_0(\mathbf{r})t}e^{-i2\pi \mathbf{k}(t)\cdot \mathbf{r}}
\end{equation}

where $\mathbf{k}(t) = \frac{1}{2\pi}\int_0^t \mathbf{G}(t')dt'$. Higher
order terms in $k$-space are very recently being considered \cite{Wilm2011}.

With this simplification our signal model now differs from a standard
FFT only in the phase accrual due to off resonance frequency,
$e^{-i\Delta \omega_0(\mathbf{r})t}$. Explicitly dealing with this
factor
still results in an unfeasibly large system matrix. However standard low rank approximations of the form
\begin{equation}
e^{-i\Delta \omega_0(\mathbf{r_j})t_i} = \sum_{l=1}^L b_{il}c_{lj}
\end{equation}
accurately caputure the field map while providing a fast algorithm for
the application of $A$.

\begin{center}
\begin{figure}
\includegraphics[width=3in]{svd_decay_image.eps}
\caption{Singular value decay for a field map dervied from a $32\times 32$ image. Images tend to be approximately piecewise constant resulting in a rapid decay of the singular values.}
\end{figure}
\end{center}


\begin{center}
\begin{figure}
\includegraphics[width=3in]{const_decay.eps}
\caption{Decay of the singular values for another image.}
\end{figure}
\end{center}

We have at our disposal many methods of approximation to the off
resonance correction. Time segmentation \cite{Sutton2003} selects a subset of points, $\{ \hat{t}_l \} \in [0,T]$ and interpolates about an optimal rate map value
\begin{equation}
b_{il} = b_l(t_i)e^{-\overline{\omega}t_i}
\end{equation}

\begin{equation}
c_{lj} = e^{-(\omega_j - \overline{\omega}t_i)\hat{t}_l}
\end{equation}

Frequency segmentation \cite{Man} selects a subset of frequencies to evaluate the exponential at and interpolates the rest.
\begin{equation}
b_{il} = e^{-\hat{\omega}_l(t_i - \overline{t})}
\end{equation}
\begin{equation}
c_{lj} = c_l(\omega_j) e^{-\omega_j\overline{t}}
\end{equation}

Standard low rank approximation methods (e.g. PCA \cite{Rokhlin},
randomized ID \cite{Liberty2007}) are also possible. These will produce
more accurate approximations to the off resonance factor at a
significant computational expense. In this work we favor the time
segmentation approach as it provides acceleration in our later
optimzations.

With this approximation our signal model becomes
\begin{eqnarray}\label{systemsum}
\mathbf{s} &=& A\mathbf{\rho} + \mathbf{\epsilon}\\
&=& (\sum_{l=1}^L \text{diag} (b_{il}) R\mathcal{F} \text{diag} (c_{lj}) )\mathbf{\rho}
\end{eqnarray}
a combination of operators satisfying the RIP.

Each term in the sum (\ref{systemsum}) has poor isometric properties as
can be seen by examiningt
\begin{eqnarray}
||B_l R\mathcal{F} C_l\mathbf{x}||_2 & \leq & ||B_l|| ||R\mathcal{F} C_l
\mathbf{x}||_2 \\
& \leq & \max_i(b_{il}) (1+\delta_{R\mathcal{F}})\max_j(c_{lj})||\mathbf{x}||_2
\end{eqnarray}
where $\delta_{R\mathcal{F}}$ is the restricted isometry constant for
the undersampled fourier operator. A similiar argument on the lower
bound yields
\begin{equation}
\delta_{B_lAC_l} \leq \min( 1-\min_i(b_{il})\min_j(c_{lj}) +
\min_i(b_{il})\min_j(c_{lj})\delta_{R\mathcal{F}}, 1-\max_i(b_{il})\max_j(c_{lj}) +
\max_i(b_{il})\max_j(c_{lj})\delta_{R\mathcal{F}}).
\end{equation}

While $\delta_{R\mathcal{F}}$ is small ($\approx 0.1$) \cite{Candes2006}
the values in $B_l$ and $C_l$ may be large depending on the field map.
Futhermore, by the triangle inequality, the RIP constant for the
complete approximation is bounded by the sum of RIP constants for each
term resulting in an encoding scheme with poor compressed sensing
opportunities.

\begin{figure}
\includegraphics{allblack.eps}
\end{figure}

\begin{figure}
\includegraphics{moreblack.eps}
\end{figure}

\begin{figure}
\includegraphics{mehblack.eps}
\end{figure}


\begin{figure}
\includegraphics{blackwhite.eps}
\end{figure}


When restricting to rank one approximations only model retains the
same RIP properties as $R\matcal{F}$.

In practive the signal equation is often underdetermined neccistiating the use of
regularizers for stable image recovery. Initial work focused on $l2$
regularization of $\nabla \rho$. While computationally efficient, $l2$
regularization in this setting has a tendency to blur edges and
introduce non-physical artifacts as the problem becomes underdetermined.

We instead recover the image by optimizing
\begin{equation}
\text{argmin} |\nabla \mathbf{\rho}|_1 \text{ s.t. } A\mathbf{\rho} +
\epsilon = \mathbf{s}.
\end{equation}

This may be solved exactly and efficiently via the Split Bregman
algorithm. The split formulation of the unconstrained step becomes
\begin{eqnarray}\label{sb}
\mathbf{\rho}^{k+1} &=& \min_\rho \frac{\mu}{2}||A\mathbf{\rho} - \mathbf{s}||^2 +\\
&& \frac{\lambda}{2}||\mathbf{d}_x^k - \nabla_x\rho - \mathbf{b}_x^k||^2 +  \frac{\lambda}{2}||\mathbf{d}_x^k - \nabla_x\rho - \mathbf{b}_x^k||^2 \\
\mathbf{d}_x^{k+1} &=& shrink(\nabla_x \mathbf{\rho}^{k+1} + \mathbf{b}_x^k, 1/\lambda)\\
\mathbf{d}_y^{k+1} &=& shrink(\nabla_y \mathbf{\rho}^{k+1} + \mathbf{b}_y^k, 1/\lambda)\\
\mathbf{b}_x^{k+1} &=& \mathbf{b}_x^{k} + (\nabla_x \mathbf{\rho}^{k+1} - \mathbf{d}_x^{k+1}) \\
\mathbf{b}_y^{k+1} &=& \mathbf{b}_y^{k} + (\nabla_y \mathbf{\rho}^{k+1} - \mathbf{d}_y^{k+1})
\end{eqnarray}


Minimization of $\mathbf{\rho}$ in (\ref{sb}) is by far the most time
consuming part of the iteration. When the system matrix is
$R\mathcal{F}$ we may exactly solve the minimization with two fourier
transforms as is done in the original Split Bregman paper.

As the model with inhomogenity correction is more sophisiticated we must  iteratively approximately solve the linear system
\begin{equation}\label{toep}
(A^TA + \lambda \triangle)\mathbf{\rho}^{k+1} = rhs^{k}.
\end{equation}

This $l2$ optimization is effectivly handled by the iterative Toeplitz
approach.





\bibliographystyle{plain}
\bibliography{thebib}

\end{document}
